\documentclass[10pt, a4paper]{article}
\usepackage[paper=a4paper, left=1.5cm, right=1.5cm, bottom=1.5cm, top=3.5cm]{geometry}
\usepackage[spanish]{babel}
\usepackage[utf8]{inputenc}
\usepackage{aed2-diseno, aed2-itef, aed2-tad, aed2-symb}
\usepackage{caratula}

\begin{document}

\titulo{Trabajo Práctico 2: "Fichin Pacalgo2"}
\materia{Algoritmos y Estructuras de Datos II}
\grupo{Grupo: tomarAgua()}

% CAMBIAR INTEGRANTES
\integrante{Reyna Maciel, Guillermo José}{393/20}{guille.j.reyna@gmail.com}
\integrante{Casado Farall, Joaquin}{072/20}{joakinfarall@gmail.com}
\integrante{Fernández Spandau, Luciana}{368/20}{fernandezspandau@gmail.com}
\integrante{Chumacero, Carlos Nehemias}{492/20}{chumacero2013@gmail.com}

\maketitle



\section{Ejercicios}

\tad{\tadNombre{direccion} es \tadNombre{string}}
\

\begin{tad}{\tadNombre{dimensión, coordenadas} \tadExtiende{\tadNombre{tupla(nat, nat)}}}

\tadOtrasOperaciones
\tadOperacion{\argumento = \argumento}{tupla(nat, nat), tupla(nat, nat)}{bool}{}
\tadOperacion{mover}{tupla(nat,nat)/$t$, direccion/$d$}{tupla(nat,nat)}{d \in \{ "arr", "abj", "der", "izq" \}}

\tadAxiomas[\paratodo{tupla(nat, nat)}{a_1, a_2}]
\tadAlinearAxiomas{mover($a$, $d$)}
\tadAxioma{$a_1$ = $a_2$}{$\pi_1(a_1) = \pi_1(a_2) \land \pi_2(a_1) = \pi_2(a_2)$}
\tadAxioma{mover($a$, $d$)}{{\IF\ $d$ = $"arr"$ THEN $<\pi_1(a_1)$, $\pi_1(a_2) + 1>$ ELSE {\IF\ $d$ = $"abj"$ THEN $<\pi_1(a_1)$, $\pi_1(a_2) - 1>$ ELSE  {\IF\ $d$ = $"izq" $ THEN $<\pi_1(a_1)$ - 1, $\pi_1(a_2)>$ ELSE $<\pi_1(a_1)$ + 1, $\pi_1(a_2)>$ FI} FI} FI}}
\end{tad}
\

\begin{tad}{\tadNombre{Mapa}}

\tadGeneros{mapa}

\tadExporta{mapa, generadores, observadores, enRango?, estaOcupado?}

\tadUsa{\tadNombre{bool}, \tadNombre{nat}, \tadNombre{conjunto($\alpha$)}, \tadNombre{coordenadas}, \tadNombre{dimensión}}
\
\tadIgualdadObservacional{m}{m'}{mapa}{dimensión($m$) \igobs dimensión($m'$) $\land$ \\ paredes($m$) \igobs paredes($m'$) $\land$ \\ fantasmas($m$) \igobs fantasmas($m'$) $\land$ \\ puntoDeSalida($m$) \igobs puntoDeSalida($m'$) $\land$ \\ puntoDeLlegada($m$) \igobs puntoDeLlegada($m'$)}
\tadAlinearFunciones{agregarFantasma}{dimension/d,coordenadas/inicio}

\tadObservadores
\tadOperacion{dimensión}{mapa}{dimensión}{}
\tadOperacion{paredes}{mapa}{conj(coordenadas)}{}
\tadOperacion{fantasmas}{mapa}{conj(coordenadas)}{}
\tadOperacion{puntoDeSalida}{mapa}{coordenadas}{}
\tadOperacion{puntoDeLlegada}{mapa}{coordenadas}{}

\tadGeneradores{}
\tadOperacion{nuevoMapa}{dimensión/d, coordenadas/inicio, coordenadas/fin} {mapa}{$\pi_1$(d) * $\pi_2$(d) $\geq$ 2 $\land$ $\neg$ (inicio = fin)
$\land$ \\ enRango?(inicio, d) $\land$ enRango?(fin, d)}
\tadOperacion{agregarFantasma}{mapa/m, coordenadas/c}{mapa}{enRango?($c$, dimensión($m$))  $\land$ $\neg$estaOcupado?(c, m)}
\tadOperacion{agregarPared}{mapa/m, coordenadas/c}{mapa}{enRango?($c$, dimensión($m$))  $\land$ $\neg$estaOcupado?(c, m)}

\tadOtrasOperaciones
\tadOperacion{esRango?}{coordenadas, dimensión}{bool}{}
\tadOperacion{estaOcupado?}{coordenadas/c, mapa/m}{bool}{enRango?($c$, dimension($m$))}

\tadAxiomas[\paratodo{mapa}{m}, \paratodo{coordenadas}{i, f, c}, \paratodo{dimensión}{d}]
\tadAlinearAxiomas{puntoDeLlegada(agregarFantasma($m$, $c$))}
\tadAxioma{dimension(nuevoMapa($d$, $i$, $f$))}{$d$}
\tadAxioma{dimension(agregarFantasma($m$, $c$))}{dimension($m$)}
\tadAxioma{dimension(agregarPared($m$, $c$))}{dimension($m$)}
\
\tadAxioma{paredes(nuevoMapa($d$, $i$, $f$))}{$\emptyset$}
\tadAxioma{paredes(agregarFantasma($m$, $c$))}{paredes($m$)}
\tadAxioma{paredes(agregarPared($m$, $c$))}{Ag($c$, paredes($m$))}
\ 
\tadAxioma{fantasmas(nuevoMapa($d$, $i$, $f$))}{$\emptyset$}
\tadAxioma{fantasmas(agregarFantasma($m$, $c$))}{Ag($c$, paredes($m$))}
\tadAxioma{fantasmas(agregarPared($m$, $c$))}{paredes($m$)}
\ 
\tadAxioma{puntoDeSalida(nuevoMapa($d$, $i$, $f$))}{$i$}
\tadAxioma{puntoDeSalida(agregarFantasma($m$, $c$))}{puntoDeSalida($m$)}
\tadAxioma{puntoDeSalida(agregarPared($m$, $c$))}{puntoDeSalida($m$)}
\ 
\tadAxioma{puntoDeLlegada(nuevoMapa($d$, $i$, $f$))}{$f$}
\tadAxioma{puntoDeLlegada(agregarFantasma($m$, $c$))}{puntoDeLlegada($m$)}
\tadAxioma{puntoDeLlegada(agregarPared($m$, $c$))}{puntoDeLlegada($m$)}
\ 
\tadAxioma{enRango?($c$, $d$)}{$\pi_1$($c$) $<$ $\pi_1$($d$) $\land$ $\pi_2$($c$) $<$ $\pi_2$($d$)}
\ 
\tadAxioma{estaOcupado?($c$, $m$)}{$\lnot(c \in$ (paredes($m$) $\cup$ fantasmas($m$) $\cup$ \{puntoDeSalida($m$), puntoDeLlegada($m$)\})}


\end{tad}

\pagebreak

\section{Aclaraciones}
\begin{itemize}

\item Estamos utilizando un sistema de coordenadas al estilo cartesiano, donde la primera coordenada representa el eje horizontal y aumenta hacia la derecha
    y la segunda coordenada representa el eje vertical y aumenta hacia arriba.
\item Por conveniencia, interpretamos que los chocolates no pueden solaparse con el punto de salida (porque para comer un
chocolate el jugador debe moverse a la coordenada en donde se encuentra el chocolate) ni con paredes. Interpretamos que sí
pueden solaparse con fantasmas y con el punto de llegada.
\item En la axiomatización del observador chocolates de partida, para simplificar la escritura
utilizamos la propiedad de que $c - \{a\} = c$ cuando $a \notin c$, pues
si la intersección de dos conjuntos es vacía, la resta no los modifica.
\item Interpretamos que los movimientos con inmunidad no son acomulativos, sino que se recargan. Es decir, comer un chocolate
mientras el jugador ya tiene inmunidad de otro chocolate no le suma 10 movimientos más de inmunidad, sino que le recarga los
movimientos con inmunidad a un máximo de 10 (similar a la funcionalidad de la inmunidad en el Pac-Man original).
\item Si bien esta especificación permite partidas inganables o imperdibles (e.g. el punto de llegada esté rodeado de paredes, o que haya
un fantasma al lado del punto de salida), consideramos que restringir estas posibilidades sería sobreespecificar, puesto
que el enunciado no las menciona.
\item Interpretamos que el punto de salida y el punto de llegada no pueden estar en el mismo espacio y que estos
no se pueden solapar con fantasmas o paredes, puesto que el enunciado dice que se asignan dos casilleros especiales para ellos.
\item En la axiomatización de inmunidadAlMover, la función máx se asume como la del TAD Natural, pero el parámetro
movimientosConInmunidad($p) - 1$ puede ser $-1$ cuando movimientosConInmunidad($p) = 0$. Asumimos que la función máximo se extiende con los enteros.
\item El enunciado dice que el puntaje se debe poder calcular al final de la partida. En nuestra especificación el puntaje
se puede calcular en cualquier punto de la partida, y en particular cuando la partida finalizó y el jugador ganó.

\end{itemize}

\end{document}
