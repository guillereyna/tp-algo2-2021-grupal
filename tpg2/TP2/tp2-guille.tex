\documentclass[10pt, a4paper]{article}
\usepackage[paper=a4paper, left=1.5cm, right=1.5cm, bottom=1.5cm, top=3.5cm]{geometry}
\usepackage[spanish]{babel}
\usepackage[utf8]{inputenc}
\usepackage{aed2-diseno, aed2-itef, aed2-tad, aed2-symb}
\usepackage{caratula}

\begin{document}

\titulo{Trabajo Práctico 2: "Fichín"}
\materia{Algoritmos y Estructuras de Datos II}
\grupo{Grupo: tomarAgua()}

% CAMBIAR INTEGRANTES
\integrante{Reyna Maciel, Guillermo José}{393/20}{guille.j.reyna@gmail.com}
\integrante{Casado Farall, Joaquin}{072/20}{joakinfarall@gmail.com}
\integrante{Fernández Spandau, Luciana}{368/20}{fernandezspandau@gmail.com}
\integrante{Chumacero, Carlos Nehemias}{492/20}{chumacero2013@gmail.com}

\maketitle

\section{TAD Fichín}

\begin{tad}{\tadNombre{Fichín}}
\tadGeneros{}
\tadExporta{}
\tadUsa{\tadNombre{}}

\tadIgualdadObservacional{f}{f'}{fichín}{}
\tadAlinearFunciones{}{}
\tadObservadores

\tadOperacion{}{}{}{}

\tadGeneradores

\tadOperacion{}{}{}{}

\tadOtrasOperaciones

\tadOperacion{}{}{}{}

\tadAxiomas[\paratodo{fichín}{f}, \paratodo{partida}{p}, \paratodo{nombre}{n}, \paratodo{nat}{m}, \paratodo{dicc(nombre, nat)}{r}]
\tadAlinearAxiomas{partidaEnCurso(iniciarPartida($f, p, n$))}
\tadAxioma{partidaEnCurso(iniciarFichin)}{false}
\tadAxioma{partidaEnCurso(iniciarPartida($f, p, n$))}{$\neg$terminóElJuego($p$)}
\tadAxioma{partidaEnCurso(moverF($f, d$))}{$\neg$terminóElJuego(mover(partida($f$), $d$))}
\
\tadAxioma{partida(iniciarPartida($f, p, n$))}{$p$}
\tadAxioma{partida(moverF($f, d$))}{mover(partida($f$), $d$)}
\
\tadAxioma{ranking(iniciarFichin)}{vacío}
\tadAxioma{ranking(iniciarPartida($f, p, n$))}{ranking($f$)}
\tadAxioma{ranking(moverF($f, d$))}{\IF ganó?(partida(moverF($f, d$))) $\land$
                                        ($\neg$jugadorEnRanking($f$) $\lor$L puntaje(partida(moverF($f, d$))) $<$ mejorPuntajeDelJugador($f$))
                                        THEN definir(nombreJugador($f$), puntaje(partida(moverF($f, d$))))
                                        ELSE ranking($f$) FI}
\ 
\tadAxioma{nombreJugador(iniciarPartida($f, p, n$))}{n}
\tadAxioma{nombreJugador(moverF($f, d$))}{nombreJugador($f$)}
\ 
\tadAxioma{mejorPuntajeDelJugador($f$)}{obtener(nombreJugador($f$), ranking($f$))}
\tadAxioma{proximoMejorPuntaje($f$)}{obtener(nombreDelProximo($f$), ranking($f$))}
\tadAxioma{nombreDelProximo($f$)}{siguienteClaveMayor(ranking($f$, mejorPuntajeDelJugador($f$)))}
\tadAxioma{siguienteClaveMayor($r, m$)}{\IF \#(claves($r$)) = 1 $\lor$ \\ ($m$ $<$ obtener(dameUno(claves($r$)), $r$) $\land$
                                             \\ obtener(dameUno(claves($r$)), $r$) $<$
                                             \\ siguienteClaveMayor(borrar(dameUno(claves($r$)), $r$)), $m$))
                                             \\ THEN dameUno(claves($r$)) ELSE siguienteClaveMayor(borrar(dameUno(claves($r$)), $r$)), $m$) FI}
\tadAxioma{jugadorEnRanking($f$)}{def?(nombreJugador($f$), ranking($f$))}
\tadAxioma{jugadorPrimero($f$)}{mayorClave(ranking($f$)) = nombreJugador($f$)}
\tadAxioma{mayorClave($r$)}{\IF \#(claves($r$)) = 1 $\lor$
                                \\ (mayorClave(sinUno($r$)) $<$ obtener(dameUno(claves($r$))))
                                \\ THEN dameUno($r$) ELSE mayorClave(sinUno($r$)) FI}

\end{tad}

\end{document}