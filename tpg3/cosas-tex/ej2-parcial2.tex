\documentclass[10pt,a4paper]{article}
\usepackage[paper=a4paper, hmargin=1.5cm, bottom=1.5cm, top=3.5cm]{geometry}
\usepackage[utf8]{inputenc}
\usepackage[spanish]{babel}
\usepackage{fancyhdr}
\usepackage{xspace}
\usepackage{xargs}
\usepackage{ifthen}
\usepackage{aed2-tad,aed2-symb,aed2-itef,aed2-diseno}
\usepackage{algorithmicx, algpseudocode, algorithm}

\pagestyle{fancy} % Esto es para el header en páginas subsecuentes%

\author{Juana Sarasa
\\ LU: 123/45 DNI: XXXXXXX}
\date{15/05/2021}
\title{Segundo Parcial - Algoritmos y Estructuras de Datos II
\\ Ejercicio 2 - Diseño}

\lhead{Juana Sarasa} % header izquierda
\rhead{15/05/2021}                  % header derecha
% para que no haya header es suficiente con sacar estos dos comandos

\begin{document}

\maketitle


%------------------------INTERFAZ--------------------------
\section{}

\begin{Interfaz}

    \textbf{parámetros formales}\hangindent=2\parindent\\
    \parbox{1.7cm}{\textbf{géneros}}$\alpha$\\
    \parbox[t]{1.7cm}{\textbf{función}}\parbox[t]{.5\textwidth-\parindent-1.7cm}{%
      \InterfazFuncion{funcionInterfaz}{\In{a}{$\alpha$}}{bool}
      [pre]%pres
      {post}%post
      [$\Theta(n)$]%complejidad
      [descripción de función]%descripcion
    }%
  
    \textbf{se explica con}: \tadNombre{TAD Representado}
  
    \textbf{géneros}: \TipoVariable{género de TAD representado}.
  
    \Titulo{Operaciones básicas de --}
  
    \InterfazFuncion{generador}{}{genero}%NOMBRE
    {$res$ $\igobs$ observador}%POST
    [$\Theta(1)$]%COMPLEJIDAD
    [genera una instancia del TAD.]%DESCRIPCION
  
    \InterfazFuncion{Definir}{\Inout{d}{dicc($\kappa,\sigma$)}, \In{k}{$\kappa$}, \In{s}{$\sigma$}}{itDicc($\kappa, \sigma$)}
    [pre]%PRE
    {post}%POST
    [$\displaystyle\Theta\left(k\right)$, donde $k = blah$]
    [descripción.]%DESCRIPCIÓN
    [aliasing.]%ALIASING
  
  \Titulo{Especificación de las operaciones auxiliares utilizadas en la interfaz}
  
      \tadOperacion{nombreOp($a$)}{parametro/p}{salida}{restriccion}
      \tadAxioma{nombreOp($a$)}{axiomas}

  \end{Interfaz}



%-----------------------------REPRESENTACION-----------------------
\section{}
\begin{Representacion}
    
    Descripción de la estructura de representación
  
    \begin{Estructura}{género}[estr]%genero%estr
      \begin{Tupla}[estr]%
        \tupItem{t}{tipo}%
        \tupItem{t2}{t2}%
      \end{Tupla}
    \end{Estructura}
  
    \Rep[estr][e]{\\%estr%e
        \textbf{en palabras:} condiciones}
  
    ~
  
    \AbsFc[estr]{genero TAD}[e]{
        cosas que tienen que pasar}

\end{Representacion}

%------------------------------ALGORITMOS------------------------
\section{}
\begin{Algoritmos}
    
    \begin{algorithm}[H]{\textbf{iDefinir}(\Inout{d}{dic}, \In{k}{$\kappa$}, \In{s}{$\sigma$}) $\to$ $res$ : $itDicc$}
        \begin{algorithmic}
            \While{condicionwhile} \Comment $\mathcal{O}(n)$
                \If{condicionif} \Comment $\mathcal{O}(1)$
                    \State $res = $ b \Comment $\mathcal{O}(1)$
                \Else
                    \State $res = $ a \Comment $\mathcal{O}(1)$
                \EndIf
            \EndWhile
            \Statex \underline{Complejidad:} $\mathcal{O}(n)$
            \Statex \underline{Justificación:} $\mathcal{O}(n)$.
        \end{algorithmic}
        \end{algorithm}


\end{Algoritmos}

%\section{Aclaraciones}
%\begin{itemize}
%    \item aclaracion
%\end{itemize}

\end{document}