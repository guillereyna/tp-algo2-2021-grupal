\documentclass[10pt,a4paper]{article}
\usepackage[paper=a4paper, hmargin=1.5cm, bottom=1.5cm, top=3.5cm]{geometry}
\usepackage[utf8]{inputenc}
\usepackage[spanish]{babel}
\usepackage{fancyhdr}
\usepackage{xspace}
\usepackage{xargs}
\usepackage{ifthen}
\usepackage{aed2-tad,aed2-symb,aed2-itef,aed2-diseno}
\usepackage{algorithmicx, algpseudocode, algorithm}
\usepackage{caratula}

\begin{document}

\titulo{Trabajo Práctico 3: "Pacalgo2"}
\materia{Algoritmos y Estructuras de Datos II}
\grupo{Grupo: tomarAgua()}

% CAMBIAR INTEGRANTES
\integrante{Reyna Maciel, Guillermo José}{393/20}{guille.j.reyna@gmail.com}
\integrante{Casado Farall, Joaquin}{072/20}{joakinfarall@gmail.com}
\integrante{Fernández Spandau, Luciana}{368/20}{fernandezspandau@gmail.com}
\integrante{Chumacero, Carlos Nehemias}{492/20}{chumacero2013@gmail.com}

\maketitle


%------------------------INTERFAZ--------------------------
\section{}

\begin{Interfaz}

    \textbf{parámetros formales}\hangindent=2\parindent\\
    \parbox{1.7cm}{\textbf{géneros}}$\alpha$\\
    \parbox[t]{1.7cm}{\textbf{función}}\parbox[t]{.5\textwidth-\parindent-1.7cm}{%
      \InterfazFuncion{funcionInterfaz}{\In{a}{$\alpha$}}{bool}
      [pre]%pres
      {post}%post
      [$\Theta(n)$]%complejidad
      [descripción de función]%descripcion
    }%
  
    \textbf{se explica con}: \tadNombre{Fichin}
  
    \textbf{géneros}: \TipoVariable{fichin}.
  
    \textbf{Operaciones básicas de fichin}
  
    %NUEVO FICHIN
    \InterfazFuncion{NuevoFichin}{\In{m}{mapa}}{fichin}%NOMBRE
    {$res$ $\igobs$ nuevoFichin(m)}%POST
    [$\Theta(c + \#paredes + \#fantasmas)$]%COMPLEJIDAD
    [genera una instancia del Fichin.]%DESCRIPCION
    
    %NUEVA PARTIDA
    \InterfazFuncion{NuevaPartida}{\Inout{f}{fichin}, \In{j}{jugador}}{}
    [f = f_0 $ \land$ $\neg$alguienJugando?(f_0)]%PRE
    {f $\igobs$ nuevaPartida(f_0, j}%POST
    [$\Theta(c)$]
    [Crea una partida para el jugador especificado, en este proceso tambien se restauran los chocolates en el mapa.]%DESCRIPCIÓN
    %ALIASING

    %MOVER
    \InterfazFuncion{Mover}{\Inout{f}{fichin}, \In{d}{direccion}}{}
    [f = f_0 $ \land$ $\neg$alguienJugando?(f_0)]%PRE
    {f $\igobs$ nuevaPartida(f_0, d}%POST
    [$\Theta(1)$ normalmente / $\Theta(|J|)$ cuando ganó o perdió.]
    [Realiza el movimiento del jugador.]%DESCRIPCIÓN
    %ALIASING
    
     %MAPA
    \InterfazFuncion{Mapa}{\In{f}{fichin}}{mapa}%NOMBRE
    {$res$ $\igobs$ mapa(f)}%POST
    [$\Theta(1)$]%COMPLEJIDAD
    [Devuelve el mapa actual.]%DESCRIPCION
    [Se devuelve una referencia inmutable.]
    
    
    %JUGADOR ACTUAL
    \InterfazFuncion{JugadorActual}{\In{f}{fichin}}{jugador}%NOMBRE
    [alguienJugando?(f)]
    {$res$ $\igobs$ jugadorActual(f)}%POST
    [$\Theta(1)$]%COMPLEJIDAD
    [Devuelve el nombre del jugador actual]%DESCRIPCION
    [Se devuelve una referencia inmutable.]
    
    
    %RANKING
    \InterfazFuncion{JugadorActual}{\In{f}{fichin}}{ranking}%NOMBRE
    {$res$ $\igobs$ ranking(f)}%POST
    [$\Theta(1)$]%COMPLEJIDAD
    [Devuelve el diccionario usado en el fichin]%DESCRIPCION
    [Se devuelve una referencia inmutable.]
    
    
    
  
  \Titulo{Especificación de las operaciones auxiliares utilizadas en la interfaz}
  
      \tadOperacion{nombreOp($a$)}{parametro/p}{salida}{restriccion}
      \tadAxioma{nombreOp($a$)}{axiomas}

  \end{Interfaz}



%-----------------------------REPRESENTACION-----------------------
\section{}
\begin{Representacion}
    
    Descripción de la estructura de representación
  
    \begin{Estructura}{fichin}[estr]%genero%estr
      \begin{Tupla}[estr]%
        \tupItem{m}{puntero(mapa)}%
        \tupItem{p}{puntero(partida}%
        \tupItem{tablero}{array(array(tupla(bool, bool, bool)))}
        \tupItem{hayAlguine}{bool}
        \tupItem{jugador}{string}
        \tupItem{ranking}{diccTrie(string, nat)}
      \end{Tupla}
    \end{Estructura}
  
  
    \Rep[estr][e]{$(1)\land(2)$ \\
    \textbf{donde:}
    \tadAxioma{(1)}{e.m = (e.p \rightarrow mapa)}
    \tadAxioma{(2)}{e.paredes $\cap$ e.fantasmas $\cap$ e.chocolates $= \emptyset$}
    \tadAxioma{(3)}{(\paratodo{nat}{i,j})(($0 \leq i < e.m.largo \land 0 \leq j < e.m.alto$) $\Rightarrow$ 
    ($\pi_0$(e.tablero[i][j]) $\Leftrightarrow$ tupla(i, j) $\in$ e.m.paredes)) $\land_L$ \\
    (\paratodo{nat}{i,j})(($0 \leq i < e.m.largo \land 0 \leq j < e.m.alto$) $\Rightarrow$ 
    ($\pi_0$(e.tablero[i][j]) $\Leftrightarrow$ tupla(i, j) $\in$ e.m.paredes))
    $\land_L$ \\
    (\paratodo{nat}{i,j})(($0 \leq i < e.m.largo \land 0 \leq j < e.m.alto$) $\Rightarrow$ 
    ($\pi_0$(e.tablero[i][j]) = $true$ $\Rightarrow$ tupla(i, j) $\in$ e.m.paredes))
    }
    \\
    \textbf{en palabras:}\\
        \textbf{m debe ser igual al mapa de p (partida)}\\
        \textbf{tablero debe ser igual al mapa (menos los chocolates que dejamos en falso)}\\
    
    }   
  
    ~
  
    \AbsFc[estr]{genero TAD}[e]{
        cosas que tienen que pasar}

\end{Representacion}

%------------------------------ALGORITMOS------------------------
\section{}
\begin{Algoritmos}
    %$\to$ $res$ : $itDicc$
    \begin{algorithm}[H]{\textbf{iNuevaPartida}(\Inout{f}{estr}, \In{j}{$string$})}
        \begin{algorithmic}
        
            \State RepoblarChocolates(f.tablero, f.m)
            \State f.p $\leftarrow$ NuevaPartida(f.m, &f.tablero)
            \State f.hayAlguien $\leftarrow$ true
            \State f.jugador $\leftarrow$ j
        
            \Statex \underline{Complejidad:} $\Theta(c)$
            \Statex \underline{Justificación:} \textbf{Cuando se repueblan los chocolates se hace en $\Theta(c)$ dado a que se recorre un arreglo con las coordenadas de estas para restaurar los chocolates.
		Esto sumado a las operaciones en $\Theta(1)$ que le siguen dan como resultado a una complejidad $\Theta(c)$}
        \end{algorithmic}
        \end{algorithm}
        
        \begin{algorithm}[H]{\textbf{iMapa}(\In{f}{estr}) $\to$ $res$ : $puntero(mapa)$}
        \begin{algorithmic}
        
            \State res $\leftarrow$ f.m
            
        
            \Statex \underline{Complejidad:} $\Theta(1)$
        \end{algorithmic}
        \end{algorithm}
        
         \begin{algorithm}[H]{\textbf{iJugadorActual}(\In{f}{estr}) $\to$ $res$ : $string$}
        \begin{algorithmic}
        
            \State res $\leftarrow$ f.jugador
            
        
            \Statex \underline{Complejidad:} $\Theta(1)$
        \end{algorithmic}
        \end{algorithm}
        
        \begin{algorithm}[H]{\textbf{iRanking}(\In{f}{estr}) $\to$ $res$ : $ranking$}
        \begin{algorithmic}
        
            \State res $\leftarrow$ f.ranking
            
        
            \Statex \underline{Complejidad:} $\Theta(1)$
        \end{algorithmic}
        \end{algorithm}


\end{Algoritmos}

%\section{Aclaraciones}
%\begin{itemize}
%    \item aclaracion
%\end{itemize}

\end{document}