\documentclass[10pt, a4paper]{article}
\usepackage[paper=a4paper, left=1.5cm, right=1.5cm, bottom=1.5cm, top=3.5cm]{geometry}
\usepackage[spanish]{babel}
\usepackage[utf8]{inputenc}
\usepackage{aed2-symb,aed2-itef,aed2-tad,aed2-diseno}


\author{Algoritmos y Estructuras de Datos II, DC, UBA.}
\date{}
\title{Super TP AED2}

\begin{document}

\maketitle

\section{Parte 1}

\subsection{TAD \tadNombre{Partida}}

\begin{tad}{\tadNombre{Partida}}
\tadGeneros{}
\tadExporta{}
\tadUsa{\tadNombre{}}

\tadIgualdadObservacional{p}{p'}{partida}{}
\tadAlinearFunciones{}{}
\tadObservadores

\tadOperacion{}{}{}{}

\tadGeneradores

\tadOperacion{}{}{}{}

\tadOtrasOperaciones

\tadOperacion{}{}{}{}

\tadAxiomas[\paratodo{partida}{p}, \paratodo{mapa}{m}, \paratodo{coordenadas}{c}, \paratodo{conj(coordenadas)}{f}]
\
\tadAlinearAxiomas{hayFantasmasCerca($\langle x, y\rangle, f$)}
\tadAxioma{jugador(iniciarPartida($m$))}{puntoDeSalida($m$)}
\tadAxioma{jugador(arriba($p$))}{$\langle\pi_1$(jugador($p$)), $\pi_2$(jugador($p$))$ + 1\rangle$}
\tadAxioma{jugador(abajo($p$))}{$\langle\pi_1$(jugador($p$)), $\pi_2$(jugador($p$))$ - 1\rangle$}
\tadAxioma{jugador(derecha($p$))}{$\langle\pi_1$(jugador($p$))$ + 1$, $\pi_2$(jugador($p$))$\rangle$}
\tadAxioma{jugador(izquierda($p$))}{$\langle\pi_1$(jugador($p$))$ - 1$, $\pi_2$(jugador($p$))$\rangle$}
\
\tadAxioma{mapa(iniciarPartida($m$))}{$m$}
\tadAxioma{mapa(arriba($p$))}{mapa($p$)}
\tadAxioma{mapa(abajo($p$))}{mapa($p$)}
\tadAxioma{mapa(izquierda($p$))}{mapa($p$)}
\tadAxioma{mapa(derecha($p$))}{mapa($p$)}
\
\tadAxioma{seAsustó?($p$)}{hayFantasmasCerca(jugador($p$), fantasmas(mapa($p$)))}

\tadAxioma{ganó?($p$)}{jugador($p$) = puntoDeLlegada(mapa($p$))}

\tadAxioma{terminóElJuego?($p$)}{seAsustó($p$) $\lor$ ganó($p$)}

\tadAxioma{hayFantasmasCerca($c, f$)}{\IF\ vacío?(f) THEN false
                            ELSE ( $|\pi_1$($c$)$-\pi_1$(dameUno($f$))$|$ + $|\pi_2$($c$)$-\pi_2$(dameUno($f$))$|< 3$ ) 
                            $\lor$ hayFantasmasCerca(sinUno($f$)) FI}

\end{tad}

\end{document}